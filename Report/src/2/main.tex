\chapter{Overview}
Firstly some assumptions about the assignment will be presented, followed by a presentation of the calendar and appointment structures including the most significant methods of each.
Lastly some shortcomings of the program will be described.

Moreover the source code in this hand--in has been commented such that every method is described.

\section{Assumptions}
This calendar implemented in Smalltalk was developed with some assumptions regarding the assignment:
\begin{itemize}
    \item A calendar does not require content, i.e.~appointment or calendars.\\
        This means that a calendar can exist without any appointments and/or sub--calendars.
    \item Methods such as: \texttt{findAppointments}, \texttt{findFirstAppointment}, and\\
        \texttt{findLastAppointment}, will find appointments in a calendar and any sub--calendars of said calendar, and so forth.
    \item Appointments represent time using objects from the standard Pharo environment.\\
        This means that conversion to and from human readable strings, such as \texttt{'1993/06/19 13:37:00'} is done by the already implemented functionality.
    \item Furthermore any functionality already present in the Pharo environments standard classes will be utilized;
        instead of implementing something twice.
    \item Removing appointments or calendars from a calendar, must retain the structure of the original calendar.\\
        This means that the original calendar must not be flattened.
    \item However, parts of a calendar, i.e.~appointments and sub--calendar, are not sorted in any way after adding or removing parts.
\end{itemize}

\section{Structure of Calendars and Appointments}
The package made during this mini--project, contains two classes: \texttt{Calendar} and \texttt{Appointment}.

Both appointments and calendars can be instantiated using special class methods, thereby creating valid objects.

\subsection{Appointments}
The \texttt{Appointment} class, inherits from the \texttt{Timespan} class, which provides functionality for defining a start and an end time, as well as detecting intersections with other timespans, i.e.~if they overlap.
On top of that the \texttt{Appointment} class itself, only has one instance variable, a title.
The most interesting methods of the \texttt{Appointment} instances are:
\begin{description}
    \item[\texttt{overlap: anAppointment}]\hfill\\
        This methods is implemented using the functionalities of the superclass \texttt{Timespan}.
        It uses the \texttt{intersects: aContender} methods as described below.
    \item[\texttt{flattenTo: aList}]\hfill\\
        This keyword method is used when flattening a calendar, and takes a list as argument.
        It the adds itself to said list, and returns the list.
    \item[\texttt{title} and \texttt{title: aTitle}]\hfill\\
        These methods retrieve and change the title of a receiver appointment instance respectively.
\end{description}

As mentioned previously the \texttt{Appointment} class inherits from the \texttt{Timespan} class.
The most used methods from \texttt{Timespan} are:
\begin{description}
    \item[\texttt{intersects: aContender}]\hfill\\
        This methods checks if the timespan \texttt{aContender} intersects with the \texttt{self} timespan.
    \item[\texttt{starting: aStartTime ending: anEndTime}]\hfill\\
        This is a class method, and is used to instantiate a new timespan with a given starting and ending.
        These two times, \texttt{aStartTime} and \texttt{anEndTime} are strings, and can be in a human readable format.
        The internals of the timespan, and the classes it uses for time keeping, then converts the given strings.
\end{description}

\subsection{Calendars}
The \texttt{Calendar} class is rather simple in that it only has two instance variables, \texttt{parts} and \texttt{title}.
However, the simplicity does not carry on to the messages understood by the \texttt{Calendar}.
This class contains most of the complexity in the program and the most significant methods are the following:
\begin{description}
    \item[\texttt{add: someContent}]\hfill\\
    This method adds content to the calendar, with \texttt{someContent} being either a collection of calendars and/or appointments, or a single calendar or appointment.
    \item[\texttt{flattenCalendar}]\hfill\\
    This method creates and returns a new calendar with the same title as self, and the same content but flattened, i.e.~no sub--calendars but their content.
    It works by calling \texttt{flattenTo: aList} on all it's children, who then adds themselves if an appointment, or does the same if a calendar.
    \item[\texttt{removeAll: aBlock}]\hfill\\
    This method is used to remove content from a calendar, it takes \texttt{aBlock}, and the removes all content which satisfies said block from the parts of the calendar.
    Note, the block, must return true if something(s) should be removed and false for all other parts.
    \item[Finding appointments]\hfill\\
    All these methods use another method \texttt{appointments} which returns a flattened list of all the calendars, and its sub--calendars appointments.
    They then use this flattened list to carry out the desired functionality.

    \texttt{findAppointments: aBlock} finds all appointments which satisfies aBlock,
    whereas \texttt{findFirstAppointment: aBlock} and \texttt{findLastAppointment: aBlock}
    sorts the results and only returns the first and last appointment respectively.
    \item[\texttt{presentAsHTMLFrom: aStartTime to: anEndTime}]\hfill\\
    This method return a string conforming to HTML, representing the calendar as an day--by--day agenda.
    The output will only contain the appointments within the range of \texttt{aStartTime} to \texttt{anEndTime}.
\end{description}

\section{Shortcomings}
A weakness of my calendar implementation is, the inability to present the calendar as anything else than an agenda in HTML.
I would have preferred a week-- or month--calendar output, but due to time constraints this was not done.

Moreover I choose to focus on the core functionality, which means that implementing a solution using Seaside was deemed out of scope.
However, in retrospect it could have been helpful to use Seaside, had I used the extra time to setup up and learn it, and it would have provided a more polished way of outputting HTML.
